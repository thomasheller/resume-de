% LaTeX source file of my resume (in German).
% Copyright 2016-2017 Thomas Heller
% 
% Based on a ModernCV template:
% Copyright 2006-2015 Xavier Danaux (xdanaux@gmail.com)
%
% This work may be distributed and/or modified under the
% conditions of the LaTeX Project Public License version 1.3c,
% available at http://www.latex-project.org/lppl/.

% 10pt/11pt/12pt, a4paper/letterpaper, sans/roman
\documentclass[11pt,a4paper,sans]{moderncv}

% classic/banking/oldstyle/fancy
\moderncvstyle{classic}

% black/blue/burgundy/green/grey/orange/purple/red
\moderncvcolor{blue}

% `\sfdefault' for the default sans serif font/`\rmdefault' for the
% default roman font/any other TeX font name
% \renewcommand{\familydefault}{\sfdefault}

\nopagenumbers{}

\usepackage[utf8]{inputenc}

\usepackage[scale=0.79]{geometry}

% width of the date column:
\setlength{\hintscolumnwidth}{2.1cm}

% force width of allocated to the name/heading
% \setlength{\makecvtitlenamewidth}{20cm}

\name{Thomas}{Heller}
\title{Software-Entwickler}
\address{Wiesenweg 4}{D-56070 Koblenz}%{Deutschland}
\phone[mobile]{+49~176~78999147}
% \phone[fixed]{}
% \phone[fax]{}
\email{thomas.m.heller@gmail.com}
\homepage{thomasheller.me}
% \social[linkedin]{thomas-m-heller}
% \social[twitter]{hellerthomas}
\social[github]{thomasheller}
% \extrainfo{}

% picture height, border thickness (can be 0pt)
\photo[64pt][0.4pt]{avatar_160px}

% \quote{}

% bibliography settings:

% numerical labels
% \makeatletter\renewcommand*{\bibliographyitemlabel}{\@biblabel{\arabic{enumiv}}}\makeatother

% heading
% \renewcommand{\refname}{Veröffentlichungen}

% multiple bibliographies in one document
% \usepackage{multibib}
% \newcites{book,misc}{{Bücher},{Sonstige}}

\newcommand{\CSharp}{C\texttt{\#}}

\begin{document}

\makecvtitle

\section{Persönliche Daten}
\cvitem{Geburtstag}{26.02.1987 in Freiburg im Breisgau}
\cvitem{Familienstand}{ledig, 1 Kind}

\section{Berufliche Erfahrung \& Ausbildung}

\subsection{Berufserfahrung}

\cventry{09/2016--heute}{Software-Entwickler}{Medvigo UG}{Koblenz}{}{\begin{itemize}
\item Entwicklung webbasierter e-Health-Lösungen in PHP/Laravel und Go
\item DevOps-Aufgaben einschließlich der Verwaltung von AWS-Cloud-Resourcen
\item Erfahrung mit WebRTC-Frameworks
\end{itemize}}

\cventry{06/2015--06/2016}{Fachinformatiker und Yoga-Lehrer}{Yoga Vidya e.V.}{Horn-Bad Meinberg}{}{\begin{itemize}
\item IT-Teamleiter (seit Februar 2016)
\item Administration der Windows-Server-/Active-Directory-/Exchange-Infrastruktur
\item Administration des Aastra-Telekommunikationsservers und der VoIP-Telefone
\item Diagnose und Behebung von Hardware-, Software- und Endanwender-Problemen
\item Entwicklung von Tools zu Parsing/Generierung von Texten und MediaWiki-Schnittstellen in Java
\item Anleiten von Meditationen
\item Assistenz in Yogastunden und Yogalehrerausbildungen
\end{itemize}}

\cventry{03/2015--04/2015}{Fachinformatiker}{Support-IT GbR}{Freiburg im Breisgau}{}{\begin{itemize}
\item IT-Support vor Ort für englisch- und deutschsprachige Kunden im Raum Freiburg
\item Administration im Windows-Server-/Active-Directory-/Exchange-Umfeld
\item Webentwicklung von Lösungen zur Zeiterfassung und Warenwirtschaft in PHP/MySQL
\end{itemize}}

\cventry{08/2013--09/2014}{Software-Entwickler (Java/Android, \CSharp{})}{SWING GmbH}{Freiburg im Breisgau}{}{\begin{itemize}
\item Entwicklung von SWING2Go in Java (Android-basierte Tabletlösung in der Pflegeinformatik)
\item Entwicklung von Webschnittstellen und Lösungen zur Datenbank-Synchronisation in Java, \CSharp{}, Google Protocol Buffers und JSON
\item Entwicklung von SWING Controlling (Software zum Finanzcontrolling für Ambulante Pflegedienste) in \CSharp{} (ASP.NET und PC)
\item 3rd-Level-Support für Endanwender und IT-Administratoren in Deutschland und Schweiz
\item Durchführung interner Schulungen für das Supportpersonal (1st und 2nd Level)
\end{itemize}}

\subsection{Duale Ausbildung}

\cventry{09/2010--07/2013}{IT-System-Kaufmann}{SWING GmbH / Max-Weber-Schule}{Freiburg im  Breisgau}{Abschluss: IHK IT-System-Kaufmann \emph{(Abschlussnote: gut)}}{\begin{itemize}
\item Prototyping von SWING2Go mit PHP, MySQL, jQuery Mobile
\item Webentwicklung von SWING2Go (Software zur mobilen Datenerfassung auf iPad) in \CSharp{} (ASP.NET)
\item Entwicklung interner Analysetools zum Finanzcontrolling in \CSharp{} (PC)
\item Redesign der Unternehmenswebsite (Implementierung in HTML, CSS, JavaScript, VTL)
\item Entwicklung des Online-Kundenportals mit Intrexx
\end{itemize}}

\subsection{Praktikum}

\cventry{05/2010--08/2010}{IT-Consultant}{SWING GmbH}{Freiburg im  Breisgau}{}{\begin{itemize}
\item Telefonische Anwenderberatung für Endkunden in Sozialen Einrichtungen in Deutschland
\item Kundenspezifische Anpassungen von Branchenlösungen der Pflegeinformatik
\item Report-Design mit Combit List \& Label
\end{itemize}}

\section{Studium und schulische Bildung}

\cventry{04/2008--04/2010}{Studium Religionswissenschaft und Evangelische Theologie}{Ruprecht-Karls-Universität}{Heidelberg}{}{\begin{itemize}
\item Einführung in die Religionswissenschaft, ihre Methodik und Geschichte
\item Einführung in die Evangelische Theologie
\item Geschichte des Urchristentums, Moraltheologie und Christliche Ethik
\item Kurse über zeitgenössische religiöse Entwicklungen (wie Patchwork-Religion)
\item Kurse über Cross-Cultural-Studies in Bezug auf die gegenseitigen Einflüsse westlicher und östlicher (speziell indischer) Religion sowie neue religiöse Bewegungen in Japan
\item Kurse über Religionspsychologie
\end{itemize}}

\cventry{09/2007--03/2008}{Studium Verfahrenstechnik}{Karlsruher Institut für Technologie}{Karlsruhe}{}{\begin{itemize}
\item Grundlagen Höhere Mathematik, Elektrotechnik, Technische Mechanik, Chemie und Werkstoffkunde
\end{itemize}}

\cventry{09/2003--07/2007}{Informationstechnisches Gymnasium}{Richard-Fehrenbach-Gewerbeschule}{Freiburg im  Breisgau}{\emph{(Abschluss: Abitur Ø 2,1)}}{\begin{itemize}
\item Objektorientierte Analyse, Design und Programmierung mit UML und Java
\item Datenbankdesign und -normalisierung, ER-Modelling, SQL
\item Mikrocontroller-Programmierung in C und ASM
\item Linux-Scripting (bash und zsh)
\item Grundlagenwissen Betriebssystemdesign und Netzwerktechnik
\end{itemize}
Allgemeinbildung in Mathematik, Physik, Chemie, Englisch, Deutsch, Französisch, Geschichte, Religion und Musik}

\section{Sprachen}
\cvitem{Deutsch}{Muttersprache}
\cvitem{Englisch}{fließend}
\cvitem{Französisch}{Grundkenntnisse}

\section{Weiteres}
\cvitem{Führerschein}{Klasse B}

\cventry{01/2015--02/2015}{Yogalehrerausbildung}{Yoga Vidya e.V.}{Horn-Bad Meinberg}{}{400 UE Intensivausbildung, Abschluss: Yogalehrer (BYV)}

\end{document}

